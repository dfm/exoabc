% Copyright 2015-2016 Dan Foreman-Mackey and the co-authors listed below.

\documentclass[manuscript, letterpaper]{aastex6}

\pdfoutput=1

\include{vc}

\usepackage{microtype}

\usepackage{url}
\usepackage{amssymb,amsmath}
\usepackage{natbib}
\usepackage{multirow}
\bibliographystyle{aasjournal}

% ----------------------------------- %
% start of AASTeX mods by DWH and DFM %
% ----------------------------------- %

\setlength{\voffset}{0in}
\setlength{\hoffset}{0in}
\setlength{\textwidth}{6in}
\setlength{\textheight}{9in}
\setlength{\headheight}{0ex}
\setlength{\headsep}{\baselinestretch\baselineskip} % this is 2 lines in ``manuscript''
\setlength{\footnotesep}{0in}
\setlength{\topmargin}{-\headsep}
\setlength{\oddsidemargin}{0.25in}
\setlength{\evensidemargin}{0.25in}

\linespread{0.54} % close to 10/13 spacing in ``manuscript''
\setlength{\parindent}{0.54\baselineskip}
\hypersetup{colorlinks = false}
\makeatletter % you know you are living your life wrong when you need to do this
\long\def\frontmatter@title@above{
\vspace*{-\headsep}\vspace*{\headheight}
\noindent\footnotesize
{\noindent\footnotesize\textsc{\@journalinfo}}\par
{\noindent\scriptsize Preprint typeset using \LaTeX\ style AASTeX6\\
With modifications by David W. Hogg and Daniel Foreman-Mackey
}\par\vspace*{-\baselineskip}\vspace*{0.625in}
}%
\makeatother

% Section spacing:
\makeatletter
\let\origsection\section
\renewcommand\section{\@ifstar{\starsection}{\nostarsection}}
\newcommand\nostarsection[1]{\sectionprelude\origsection{#1}}
\newcommand\starsection[1]{\sectionprelude\origsection*{#1}}
\newcommand\sectionprelude{\vspace{1em}}
\let\origsubsection\subsection
\renewcommand\subsection{\@ifstar{\starsubsection}{\nostarsubsection}}
\newcommand\nostarsubsection[1]{\subsectionprelude\origsubsection{#1}}
\newcommand\starsubsection[1]{\subsectionprelude\origsubsection*{#1}}
\newcommand\subsectionprelude{\vspace{1em}}
\makeatother

\widowpenalty=10000
\clubpenalty=10000

\sloppy\sloppypar

% ------------------ %
% end of AASTeX mods %
% ------------------ %

% Projects:
\newcommand{\project}[1]{\textsl{#1}}
\newcommand{\kepler}{\project{Kepler}}
\newcommand{\KT}{\project{K2}}
\newcommand{\tess}{\project{TESS}}
\newcommand{\plato}{\project{PLATO}}
\newcommand{\gaia}{\project{Gaia}}
\newcommand{\pdc}{\project{PDC}}
\newcommand{\bls}{\project{BLS}}
\newcommand{\emcee}{\project{emcee}}
\newcommand{\exosyspop}{\project{exosyspop}}

\newcommand{\foreign}[1]{\emph{#1}}
\newcommand{\etal}{\foreign{et\,al.}}
\newcommand{\etc}{\foreign{etc.}}
\newcommand{\True}{\foreign{True}}
\newcommand{\Truth}{\foreign{Truth}}

\newcommand{\dfmfigref}[1]{\ref{fig:#1}}
\newcommand{\dfmFig}[1]{Figure~\dfmfigref{#1}}
\newcommand{\dfmfig}[1]{\dfmFig{#1}}
\newcommand{\dfmfiglabel}[1]{\label{fig:#1}}

% \newcommand{\Tab}[1]{Table~\ref{tab:#1}}
% \newcommand{\tab}[1]{\Tab{#1}}
\newcommand{\tablabel}[1]{\label{tab:#1}}

\renewcommand{\eqref}[1]{\ref{eq:#1}}
\newcommand{\Eq}[1]{Equation~(\eqref{#1})}
\newcommand{\eq}[1]{\Eq{#1}}
\newcommand{\eqalt}[1]{Equation~\eqref{#1}}
\newcommand{\eqlabel}[1]{\label{eq:#1}}

\newcommand{\sectionname}{Section}
\newcommand{\sectref}[1]{\ref{sect:#1}}
\newcommand{\Sect}[1]{\sectionname~\sectref{#1}}
\newcommand{\sect}[1]{\Sect{#1}}
\newcommand{\sectalt}[1]{\sectref{#1}}
\newcommand{\App}[1]{Appendix~\sectref{#1}}
\newcommand{\app}[1]{\App{#1}}
\newcommand{\sectlabel}[1]{\label{sect:#1}}

\newcommand{\T}{\ensuremath{\mathrm{T}}}
\newcommand{\dd}{\ensuremath{\,\mathrm{d}}}
\newcommand{\unit}[1]{{\ensuremath{\,\mathrm{#1}}}}
\newcommand{\bvec}[1]{{\ensuremath{\boldsymbol{#1}}}}
\newcommand{\appropto}{\mathrel{\vcenter{
  \offinterlineskip\halign{\hfil$##$\cr
    \propto\cr\noalign{\kern2pt}\sim\cr\noalign{\kern-2pt}}}}}
\newcommand{\densityunit}{{\ensuremath{\mathrm{nat}^{-2}}}}


% TO DOS
\newcommand{\todo}[3]{{\color{#2}\emph{#1}: #3}}
\newcommand{\dfmtodo}[1]{\todo{DFM}{red}{#1}}
\newcommand{\tdmtodo}[1]{\todo{TDM}{blue}{#1}}
\newcommand{\hoggtodo}[1]{\todo{HOGG}{blue}{#1}}

% Response
\newcommand{\response}[1]{{\color{blue}#1}}


% Helpers for this paper:
\newcommand{\license}{MIT License}
\newcommand{\paper}{paper}

% Notation for this paper:
\newcommand{\meanpars}{{\ensuremath{\bvec{\theta}}}}
\newcommand{\kernpars}{{\ensuremath{\bvec{\alpha}}}}
\newcommand{\params}{{\ensuremath{\bvec{w}}}}
\newcommand{\poppars}{{\ensuremath{\bvec{\beta}}}}
\newcommand{\rate}{{\ensuremath{\Gamma}}}

\newcommand{\modelname}[1]{{\textsf{#1}}}
\newcommand{\datareleaseurl}{\url{http://dx.doi.org/10.5281/zenodo.58273}}


% \shorttitle{The population of long-period transiting exoplanets}
% \shortauthors{Foreman-Mackey, Morton, Hogg, \etal}
% \submitted{Submitted to \textit{The Astrophysical Journal}}

\begin{document}

\title{%
\vspace{\baselineskip}
Approximate Bayesian Computation of the population of exoplanets
\vspace{-2\baselineskip}  % OMG AASTEX6 IS SO BROKEN
}

\newcounter{affilcounter}
\altaffiltext{1}{\textsf{danfm@uw.edu}; Sagan Fellow}

\setcounter{affilcounter}{2}
\edef \uw {\arabic{affilcounter}}\stepcounter{affilcounter}
\altaffiltext{\uw}       {Astronomy Department, University of Washington,
                          Seattle, WA, 98195, USA}

\author{%
    Daniel~Foreman-Mackey\altaffilmark{1,\uw},
    and friends
    \vspace{\baselineskip}
}


\begin{abstract}

Sick.

\end{abstract}

\keywords{%
methods: data analysis
---
methods: statistical
---
catalogs
---
planetary systems
---
stars: statistics
}

\clearpage
\section{Introduction}


\vspace{1.5em}
All of the code used in this project is available from
\url{https://github.com/dfm/exoabc} under the MIT open-source software
license.
This code (plus some dependencies) can be run to re-generate all of the
figures and results in this \paper; this version of the paper was generated
with git commit \texttt{\githash} (\gitdate).

% \acknowledgments
\vspace{1.5em}
It is a pleasure to thank
everyone
for helpful discussions and contributions.

This research made use of the NASA \project{Astrophysics Data System} and the
NASA Exoplanet Archive.
The Exoplanet Archive is operated by the California Institute of Technology,
under contract with NASA under the Exoplanet Exploration Program.

This \paper\ includes data collected by the \kepler\ mission. Funding for the
\kepler\ mission is provided by the NASA Science Mission directorate.
We are grateful to the entire \kepler\ team, past and present.
Their tireless efforts were all essential to the tremendous success of the
mission and the successes of \KT, present and future.

These data were obtained from the Mikulski Archive for Space Telescopes
(MAST).
STScI is operated by the Association of Universities for Research in
Astronomy, Inc., under NASA contract NAS5-26555.
Support for MAST is provided by the NASA Office of Space Science via grant
NNX13AC07G and by other grants and contracts.

Computing resources were provided by High Performance Computing at New York
University.

\facility{Kepler}
\software{%
    \project{batman} \citep{Kreidberg:2015},
    \project{ceres} \citep{Agarwal:2016},
    \project{corner.py} \citep{Foreman-Mackey:2016},
    \project{emcee} \citep{Foreman-Mackey:2013},
    \project{george} \citep{Ambikasaran:2016},
    \project{isochrones} \citep{Morton:2015},
	\project{matplotlib} \citep{Hunter:2007},
	\project{numpy} \citep{Van-Der-Walt:2011},
	\project{scipy} \citep{Jones:2001},
    \project{vespa} \citep{Morton:2015b}}.

% \newpage
% \appendix
% \section{Appendix}


\bibliography{exoabc}

\end{document}
